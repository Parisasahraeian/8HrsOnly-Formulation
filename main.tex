\documentclass[11pt]{article}
%
% My general LaTeX preamble
%
\usepackage[utf8]{inputenc}
\usepackage[letterpaper,margin=1in]{geometry}
\usepackage{amsmath}
\usepackage{amssymb}
\usepackage{amsthm}
\usepackage{mathtools}
\usepackage{natbib}
\usepackage{graphicx}
\usepackage{color}
\newcommand{\alert}{\textcolor{red}}
\newcommand{\comm}{\textcolor{blue}}
\newcommand{\review}{\textcolor{magenta}}
\usepackage[pdfstartview=FitH,pdfpagelayout=OneColumn]{hyperref}
\usepackage{url}
\usepackage{lineno}
\modulolinenumbers[3]
\usepackage{tabularx,multicol,multirow}
\usepackage{enumerate}
\usepackage{setspace}
\usepackage{soul} 
\usepackage{wrapfig}
%\usepackage{pslatex} %times new roman
%\usepackage{times} %times
\usepackage{palatino}
\usepackage[sc]{mathpazo} %math to match palatino
\usepackage[T1]{fontenc}


\usepackage{algorithmicx,algpseudocode}

\usepackage{tikz,tikz-network}

\linespread{1.05}
%%%%%%%%%%%%%%%%%%%%%%%%%%%%%
\theoremstyle{plain}
\newtheorem{thm}{Theorem}
\newtheorem{lem}{Lemma}
\newtheorem{prop}{Proposition}
\newtheorem{cor}{Corollary}
 \newtheorem{clm}[thm]{Claim}
 
\theoremstyle{definition}
\newtheorem{defn}{Definition}
\newtheorem{eg}{Example}
\newtheorem{form}{Formulation}

\theoremstyle{remark}
\newtheorem{rmk}{Remark}
\newtheorem{nt}{Note}

%%%%%%%%%%%%%%%%%%%%%%%%%%%%%%
\newcommand{\calP}{\mathcal{P}}
\newcommand{\calT}{\mathcal{T}}
\newcommand{\calG}{\mathcal{G}}
\newcommand{\calB}{\mathcal{B}}
\newcommand{\calL}{\mathcal{L}}
\newcommand{\calU}{\mathcal{U}}
\newcommand{\calN}{\mathcal{N}}
\newcommand{\rln}{\mathbb{R}^n}
\newcommand{\rlnp}{\mathbb{R}^n_+}
\newcommand{\rlnbyp}{\mathbb{R}^n\times\mathbb{R}^p}
\newcommand{\rlp}{\mathbb{R}_+}
\newcommand{\rl}{\mathbb{R}}
\newcommand{\eps}{\epsilon}
\newcommand{\lam}{\lambda}
\newcommand{\doh}{\partial}
\newcommand{\lra}{\longrightarrow}
\newcommand{\ra}{\rightarrow}
\newcommand{\imp}{\Rightarrow}
\newcommand{\grad}{\nabla}
\newcommand{\hess}{{\nabla^2}}
\newcommand{\half}{\frac{1}{2}}
\newcommand{\third}{\frac{1}{3}}
\newcommand{\twothird}{\frac{2}{3}}
\newcommand{\znp}{\mathbb{Z}^n_+}
\newcommand{\zn}{\mathbb{Z}^n}
\newcommand{\z}{\mathbb{Z}}
\newcommand{\zp}{\mathbb{Z}_+}
\newcommand{\andint}{\textrm{ and integer}}
\newcommand{\bin}{\{0,1\}^n}
\newcommand{\noun}[1]{\textsc{#1}}
\newcommand{\tends}{\rightarrow}
\newcommand{\bij}{\leftrightarrow}
\newcommand{\symdif}{\,\triangle\,}
\newcommand{\suchthat}{\;\ifnum\currentgrouptype=16 \middle\fi|\;}

\DeclarePairedDelimiter\ceil{\lceil}{\rceil}
\DeclarePairedDelimiter\floor{\lfloor}{\rfloor}
\DeclarePairedDelimiter\abs{\lvert}{\rvert}%
\DeclarePairedDelimiter\norm{\lVert}{\rVert}%
\providecommand\iff{\DOTSB\;\Longleftrightarrow\;}
\providecommand\implies{\DOTSB\;\Longrightarrow\;}
\providecommand\impliedby{\DOTSB\;\Longleftarrow\;}

% Swap the definition of \abs* and \norm*, so that \abs
% and \norm resizes the size of the brackets, and the 
% starred version does not.
\makeatletter
\let\oldabs\abs
\def\abs{\@ifstar{\oldabs}{\oldabs*}}
%
\let\oldnorm\norm
\def\norm{\@ifstar{\oldnorm}{\oldnorm*}}
\makeatother


\DeclareMathOperator{\rank}{rank}
\DeclareMathOperator*{\argmin}{argmin}
\DeclareMathOperator*{\diam}{diam}
\newcommand\Bl[1]{\mathopen{\hbox{\Large\rm#1}}}
\newcommand\Br[1]{\mathclose{\hbox{\Large\rm#1}}}
\newcommand{\cev}[1]{\reflectbox{\ensuremath{\vec{\reflectbox{\ensuremath{#1}}}}}}
\def\bx{{\bf x}}
\def\by{{\bf y}}
\def\bz{{\bf z}}
\def\bw{{\bf w}}
\def\bm{{\bf m}}
\def\bV{{\bf V}}
\def\one{{\bf 1}}
\def\zero{{\bf 0}}
\def\bull{\par $\,\bullet\,$}
\def\reals{{I\kern-.35em R}}
\def\argmin{\mathop{\rm argmin}}

\newcommand{\blue}{\textcolor{blue}}
\newcommand{\red}{\textcolor{red}}
\newcommand{\green}{\textcolor{green}}
\newcommand{\cyan}{\textcolor{cyan}}
\newcommand{\orange}{\textcolor{orange}}
\newcommand{\magenta}{\textcolor{magenta}}
\newcommand{\violet}{\textcolor{violet}}
\newcommand{\gray}{\textcolor{gray}}
\newcommand{\brown}{\textcolor{brown}}




%\usepackage{authblk}
%\author[1]{Don Joe}
%\author[2]{Smith K.}
%\author[1]{Wanderer}
%\author[1]{Static}
%\affil[1]{TeX.SX}
%\affil[2]{Both on a bus}
%\setcounter{Maxaffil}{0}
%\renewcommand\Affilfont{\itshape\small}






\begin{document}
	 
	 I want to cite a paper here~\cite{BBFNFMP2019closecentralclq}.
	 
	 Lookup: eqref; ref; label; cite commands and spacing commands in \LaTeX.

\section{Parameters}
\begin{itemize}
\item $T$ denotes the number of time slots in the planning horizon. \hl{For example, if each time slot has 15 minutes and the planning horizon is two weeks, so we have $T = 4\times 24 \times 14 = 1344$ slots in total.}

\item Number of shifts needed to cover each slot $k$ is given by $r_k$, where $k \in [T]$.

\item All shifts have equal duration, denoted by $d$.

\item Each schedule consists of $n$ shifts.

\item Number of schedules constructed is denoted by $m$.

\item Minimum delay between two consecutive shifts in any schedule is denoted by $\Delta$. 

\end{itemize}

\section{Assumptions} 
\begin{itemize}

\item We consider only one shift type.

\item All schedules created must begin and end inside the planning horizon \underline{excluding} the minimum resting delay $\Delta$ after the last shift, i.e. no schedule or overtime shift can span across two different pay-periods. Since the minimum resting delay after the last shift cannot be enforced inside the planning horizon, this must be accounted for when schedules are assigned to employees. 

\item A shift can overrun, but it can not exceed $d$.

\end{itemize}


% \section{Shift distribution model}

% \paragraph{Decision variables}
% \indent\\

% $x_i$: Number of shifts that must start in time-slot $i$, for each $i = 1,\ldots, T-d+1$.
% \\


% \begin{align}
% \label{eq.SDobj}\textbf{(8 hr SDO)}\quad\min \sum_{i=1}^{T-d+1}x_i & \\
% \label{eq.SDcon} \text{subject to.}  \quad  \sum_{i=\max\{1,k-d+1\}}^{\min\{k,T-d+1\}}x_i&\ge r_k \quad \forall k =1,\ldots,T\\
% x_i & \ge 0 \text{ and integer } \forall  i = 1,\ldots, T-d+1
% \end{align}



\section{Shift scheduling formulation}
\begin{itemize}
    \item Let's remove $o_u$ variable. No overtime.
    \item Each shift is not fixed/equal duration; minimum shift length is $d_\ell$ and maximum shift length is $d_M$. 
\end{itemize}
\paragraph{Decision variables} 
\begin{itemize}
\item $s_{ij}$: Start time of the $i$-th shift in the $j$-th schedule, for $i = 1,\ldots,n; j=1,\ldots,m$. 
\item  $t_{ij}$ : Completion time of the $i$-th shift in the $j$-th schedule, for $i = 1,\ldots,n; j=1,\ldots,m$.
\item $z_{ijk}$ : Binary  variable that indicates if the $i$-th shift in the $j$-th schedule covers time-slot $k$, for $ i = 1,\ldots,n; j=1,\ldots,m; k=1\ldots,T$.
\item $o_{u}$: Number of overtime shifts starting at time-slot $u$, for $u= 1,\ldots, T-d+1$.
\end{itemize}
\begin{align}
\label{eq.ATCSobj}\textbf{(SSO)}\quad\min \sum_{i=1}^{n}\sum_{j=1}^{m}(t_{ij}-s_{ij}) &+ \sum_{u=1}^{T-d+1} o_{u}\\
\label{eq.ATCSnextshiftcon}\text{subject to.} \quad s_{i+1,j} \ge t_{ij} + \Delta &\quad \forall i=1,\ldots,n;\\
\nonumber &\quad\  j=1,\ldots, m\\
\label{eq.ATCSminshiftlength}t_{ij} - s_{ij} = d  &\quad \forall i=1,\ldots,n; \\
\nonumber &\quad\ j=1,\ldots, m\\
\label{eq.ATCSdemandcon} \sum_{i=1}^{n}\sum_{j=1}^{m}z_{ijk} + \sum_{u=1}^{T-d+1}o_{u}\ge r_k  &\quad \forall k=1,\ldots,T\\
\label{eq.ATCSstartb4k}s_{ij} \le k\times z_{ijk} + [T-(n-i+1)(d + \Delta)]\times(1-z_{ijk}) &\quad \forall i=1,\ldots,n; \\
\nonumber &\quad \ j=1,\ldots, m k=1,\ldots,T\\
\label{eq.ATCSendafterk}t_{ij} \ge k\times z_{ijk} + i\times(d+\Delta)\times(1-z_{ijk}) &\quad \forall  i=1,\ldots,n; \\
\nonumber &\quad \ j=1,\ldots, m; k=1,\ldots,T\\
\label{eq.ATCSdisaggZ}\sum_{i=1}^{n}z_{ijk} \le 1 &\quad  \forall  j=1,\ldots, m; \\[-3mm]
\nonumber &\quad\ k=1,\ldots,T\\
\label{eq.lastshiftbound} t_{n,j}\le T &\quad \forall  j=1,\ldots, m\\
\label{eq.ATCSNN} s_{ij}, t_{ij}  \ge 0 \text{ and integer}  &\quad \forall  i=1,\ldots,n; \\
\nonumber & \quad\ j=1,\ldots, m
\end{align}



\bibliography{refs}
% \bibliographystyle{plainnat}
\bibliographystyle{plain}
\end{document}
